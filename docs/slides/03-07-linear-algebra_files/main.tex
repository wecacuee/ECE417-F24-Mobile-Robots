\documentclass[times,t]{beamer}
\usepackage{amssymb}
\usepackage{amsmath}
\usepackage{amsfonts}
\usepackage{lmodern} 
\input{sym.tex}
\setbeamertemplate{navigation symbols}{}

\title{ECE 417/598: Singular Value Decompsition }
\author{Vikas Dhiman.  }
\date{March 7, 2022}
\begin{document}

\newcommand{\ubfu}{\underline{\bfu}}
\begin{frame}
  \titlepage
  \end{frame}
\begin{frame}
  \includegraphics[width=\linewidth]{media/lane-from-points.pdf}
  \begin{align*}
    \ubfu_1 &= [100, 98, 1]^\top\\
    \ubfu_2 &= [105, 95, 1]^\top\\
    \ubfu_3 &= [107, 90, 1]^\top\\
    \ubfu_4 &= [110, 85, 1]^\top
    \end{align*}
    Find  the line $\bfl$ such that it is the ``closest line'' passing through
    $\bfu_1, \dots, \bfu_4$.
\end{frame}

\begin{frame}
  \[
  U = \begin{bmatrix}\bfu_1^\top  \\
    \bfu_2^\top \\
    \bfu_3^\top \\
    \bfu_4^\top
  \end{bmatrix}
  \]
  We want to solve for $\bfl$ such that
  \[
    U \bfl = 0
  \]
\end{frame}

\begin{frame}
  If the eigenvectors $x_1, \dots, x_k$ correspond to different eigenvalues
  $\lambda_1, \dots, \lambda_k$ then those eigenvectors are linearly independent.
\end{frame}

\begin{frame}{Hierarchy of transforms}
  \includegraphics[width=\linewidth]{media/hierarchy-of-transforms.png}
\end{frame}

\begin{frame}{Conjugate (or Hermitian) Transpose}
\end{frame}

\begin{frame}{Hermitian or Symmetric matrices}
\end{frame}

\begin{frame}
    Property 1: If $A = A^H$ , then for all complex vectors $x \in \bbC^n$, the number $x^HAx$ is real.
\end{frame}
\begin{frame}
    Property 2: Every eigenvalue of a Hermitian matrix is real.
\end{frame}

\begin{frame}
  Property 3: The eigenvectors of a real symmetric matrix or a Hermitian matrix, if
  they come from different eigenvalues, are orthogonal to one another.
  \\
  If $A = A^T$ , the diagonalizing matrix S can be an orthogonal matrix $S^{-1}
  = S^T$ if they come from different eigenvalues.
\end{frame}




\begin{frame}{Singular Value  Decomposition (SVD)}
  \begin{align}
    A  &=   U  \begin{bmatrix}\Sigma   &  0  \\   0  &  0 \end{bmatrix} V^\top \\
    A^\top A &= V \Sigma^2  V^{-1} \\
    A^\top A \bfv_i  &= \lambda_i \bfv_i & \lambda_i = \sigma_i^2\\
    A V   &= U \begin{bmatrix}\Sigma   &  0  \\   0  &  0 \end{bmatrix}\\
    U^+   &=  \Sigma^{-1}AV^+
    \end{align}
\end{frame}

\begin{frame}{Numerical example}
  Find singular value decomposition
  \[
    A   =  \begin{bmatrix}
      1 &   2  &  3  \\
      4 & 5 &   6 \\
      2 & 4 &   6
    \end{bmatrix}
  \]
\end{frame}



\end{document}