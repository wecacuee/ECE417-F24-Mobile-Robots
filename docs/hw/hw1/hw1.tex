\documentclass[twocolumn]{article}
\usepackage[utf8]{inputenc}
\usepackage[top=1in]{geometry}
\usepackage{graphicx}
\usepackage{hyperref}
\usepackage{amsmath}
\usepackage{verbatim}
% Calligraphic fonts
\newcommand{\calA}{{\cal A}}
\newcommand{\calB}{{\cal B}}
\newcommand{\calC}{{\cal C}}
\newcommand{\calD}{{\cal D}}
\newcommand{\calE}{{\cal E}}
\newcommand{\calF}{{\cal F}}
\newcommand{\calG}{{\cal G}}
\newcommand{\calH}{{\cal H}}
\newcommand{\calI}{{\cal I}}
\newcommand{\calJ}{{\cal J}}
\newcommand{\calK}{{\cal K}}
\newcommand{\calL}{{\cal L}}
\newcommand{\calM}{{\cal M}}
\newcommand{\calN}{{\cal N}}
\newcommand{\calO}{{\cal O}}
\newcommand{\calP}{{\cal P}}
\newcommand{\calQ}{{\cal Q}}
\newcommand{\calR}{{\cal R}}
\newcommand{\calS}{{\cal S}}
\newcommand{\calT}{{\cal T}}
\newcommand{\calU}{{\cal U}}
\newcommand{\calV}{{\cal V}}
\newcommand{\calW}{{\cal W}}
\newcommand{\calX}{{\cal X}}
\newcommand{\calY}{{\cal Y}}
\newcommand{\calZ}{{\cal Z}}

% Sets:
\newcommand{\setA}{\textsf{A}}
\newcommand{\setB}{\textsf{B}}
\newcommand{\setC}{\textsf{C}}
\newcommand{\setD}{\textsf{D}}
\newcommand{\setE}{\textsf{E}}
\newcommand{\setF}{\textsf{F}}
\newcommand{\setG}{\textsf{G}}
\newcommand{\setH}{\textsf{H}}
\newcommand{\setI}{\textsf{I}}
\newcommand{\setJ}{\textsf{J}}
\newcommand{\setK}{\textsf{K}}
\newcommand{\setL}{\textsf{L}}
\newcommand{\setM}{\textsf{M}}
\newcommand{\setN}{\textsf{N}}
\newcommand{\setO}{\textsf{O}}
\newcommand{\setP}{\textsf{P}}
\newcommand{\setQ}{\textsf{Q}}
\newcommand{\setR}{\textsf{R}}
\newcommand{\setS}{\textsf{S}}
\newcommand{\setT}{\textsf{T}}
\newcommand{\setU}{\textsf{U}}
\newcommand{\setV}{\textsf{V}}
\newcommand{\setW}{\textsf{W}}
\newcommand{\setX}{\textsf{X}}
\newcommand{\setY}{\textsf{Y}}
\newcommand{\setZ}{\textsf{Z}}

% Vectors
\newcommand{\bfa}{\mathbf{a}}
\newcommand{\bfb}{\mathbf{b}}
\newcommand{\bfc}{\mathbf{c}}
\newcommand{\bfd}{\mathbf{d}}
\newcommand{\bfe}{\mathbf{e}}
\newcommand{\bff}{\mathbf{f}}
\newcommand{\bfg}{\mathbf{g}}
\newcommand{\bfh}{\mathbf{h}}
\newcommand{\bfi}{\mathbf{i}}
\newcommand{\bfj}{\mathbf{j}}
\newcommand{\bfk}{\mathbf{k}}
\newcommand{\bfl}{\mathbf{l}}
\newcommand{\bfm}{\mathbf{m}}
\newcommand{\bfn}{\mathbf{n}}
\newcommand{\bfo}{\mathbf{o}}
\newcommand{\bfp}{\mathbf{p}}
\newcommand{\bfq}{\mathbf{q}}
\newcommand{\bfr}{\mathbf{r}}
\newcommand{\bfs}{\mathbf{s}}
\newcommand{\bft}{\mathbf{t}}
\newcommand{\bfu}{\mathbf{u}}
\newcommand{\bfv}{\mathbf{v}}
\newcommand{\bfw}{\mathbf{w}}
\newcommand{\bfx}{\mathbf{x}}
\newcommand{\bfy}{\mathbf{y}}
\newcommand{\bfz}{\mathbf{z}}


\newcommand{\bfalpha}{\boldsymbol{\alpha}}
\newcommand{\bfbeta}{\boldsymbol{\beta}}
\newcommand{\bfgamma}{\boldsymbol{\gamma}}
\newcommand{\bfdelta}{\boldsymbol{\delta}}
\newcommand{\bfepsilon}{\boldsymbol{\epsilon}}
\newcommand{\bfzeta}{\boldsymbol{\zeta}}
\newcommand{\bfeta}{\boldsymbol{\eta}}
\newcommand{\bftheta}{\boldsymbol{\theta}}
\newcommand{\bfiota}{\boldsymbol{\iota}}
\newcommand{\bfkappa}{\boldsymbol{\kappa}}
\newcommand{\bflambda}{\boldsymbol{\lambda}}
\newcommand{\bfmu}{\boldsymbol{\mu}}
\newcommand{\bfnu}{\boldsymbol{\nu}}
\newcommand{\bfomicron}{\boldsymbol{\omicron}}
\newcommand{\bfpi}{\boldsymbol{\pi}}
\newcommand{\bfrho}{\boldsymbol{\rho}}
\newcommand{\bfsigma}{\boldsymbol{\sigma}}
\newcommand{\bftau}{\boldsymbol{\tau}}
\newcommand{\bfupsilon}{\boldsymbol{\upsilon}}
\newcommand{\bfphi}{\boldsymbol{\phi}}
\newcommand{\bfchi}{\boldsymbol{\chi}}
\newcommand{\bfpsi}{\boldsymbol{\psi}}
\newcommand{\bfomega}{\boldsymbol{\omega}}
\newcommand{\bfxi}{\boldsymbol{\xi}}
\newcommand{\bfell}{\boldsymbol{\ell}}

% Matrices
\newcommand{\bfA}{\mathbf{A}}
\newcommand{\bfB}{\mathbf{B}}
\newcommand{\bfC}{\mathbf{C}}
\newcommand{\bfD}{\mathbf{D}}
\newcommand{\bfE}{\mathbf{E}}
\newcommand{\bfF}{\mathbf{F}}
\newcommand{\bfG}{\mathbf{G}}
\newcommand{\bfH}{\mathbf{H}}
\newcommand{\bfI}{\mathbf{I}}
\newcommand{\bfJ}{\mathbf{J}}
\newcommand{\bfK}{\mathbf{K}}
\newcommand{\bfL}{\mathbf{L}}
\newcommand{\bfM}{\mathbf{M}}
\newcommand{\bfN}{\mathbf{N}}
\newcommand{\bfO}{\mathbf{O}}
\newcommand{\bfP}{\mathbf{P}}
\newcommand{\bfQ}{\mathbf{Q}}
\newcommand{\bfR}{\mathbf{R}}
\newcommand{\bfS}{\mathbf{S}}
\newcommand{\bfT}{\mathbf{T}}
\newcommand{\bfU}{\mathbf{U}}
\newcommand{\bfV}{\mathbf{V}}
\newcommand{\bfW}{\mathbf{W}}
\newcommand{\bfX}{\mathbf{X}}
\newcommand{\bfY}{\mathbf{Y}}
\newcommand{\bfZ}{\mathbf{Z}}


\newcommand{\bfGamma}{\boldsymbol{\Gamma}}
\newcommand{\bfDelta}{\boldsymbol{\Delta}}
\newcommand{\bfTheta}{\boldsymbol{\Theta}}
\newcommand{\bfLambda}{\boldsymbol{\Lambda}}
\newcommand{\bfPi}{\boldsymbol{\Pi}}
\newcommand{\bfSigma}{\boldsymbol{\Sigma}}
\newcommand{\bfUpsilon}{\boldsymbol{\Upsilon}}
\newcommand{\bfPhi}{\boldsymbol{\Phi}}
\newcommand{\bfPsi}{\boldsymbol{\Psi}}
\newcommand{\bfOmega}{\boldsymbol{\Omega}}


% Blackboard Bold:
\newcommand{\bbA}{\mathbb{A}}
\newcommand{\bbB}{\mathbb{B}}
\newcommand{\bbC}{\mathbb{C}}
\newcommand{\bbD}{\mathbb{D}}
\newcommand{\bbE}{\mathbb{E}}
\newcommand{\bbF}{\mathbb{F}}
\newcommand{\bbG}{\mathbb{G}}
\newcommand{\bbH}{\mathbb{H}}
\newcommand{\bbI}{\mathbb{I}}
\newcommand{\bbJ}{\mathbb{J}}
\newcommand{\bbK}{\mathbb{K}}
\newcommand{\bbL}{\mathbb{L}}
\newcommand{\bbM}{\mathbb{M}}
\newcommand{\bbN}{\mathbb{N}}
\newcommand{\bbO}{\mathbb{O}}
\newcommand{\bbP}{\mathbb{P}}
\newcommand{\bbQ}{\mathbb{Q}}
\newcommand{\bbR}{\mathbb{R}}
\newcommand{\bbS}{\mathbb{S}}
\newcommand{\bbT}{\mathbb{T}}
\newcommand{\bbU}{\mathbb{U}}
\newcommand{\bbV}{\mathbb{V}}
\newcommand{\bbW}{\mathbb{W}}
\newcommand{\bbX}{\mathbb{X}}
\newcommand{\bbY}{\mathbb{Y}}
\newcommand{\bbZ}{\mathbb{Z}}




\title{ECE 417/598: Homework 1}
\author{Max marks: 100}
\date{Due on Jan 28, 2021, 11:59 PM}
\newtheorem{prob}{Problem}
\newif\ifsol
\solfalse

\newcommand{\bx}{\bar{x}}
\newcommand{\by}{\bar{y}}
\newcommand{\bz}{\bar{z}}
\DeclareMathOperator{\arctantwo}{arctan2}
\begin{document}

\maketitle

You are allowed to use any matrix or linear algebra library (Eigen or xtensor), but no
library that implements rotation matrices. You are not allowed to use
Eigen/Geometry. You can use the following code for generating random rotation matrices:
\href{https://github.com/wecacuee/ECE417-Mobile-Robots/blob/master/notebooks/random_rotation.cpp}{random\_rotation.cpp}.
You can also use the following template to fill in your answers: \href{https://vikasdhiman.info/ECE417-Mobile-Robots/hw/hw1/code/hw1.cpp}{hw1.cpp}

\section{Jan 24 Lecture: 3D transformations}
\begin{prob}
  Degrees of Freedom of a quantity is the number independent scalar variables
  needed to represent that quantity. What is degrees of freedom required to 
  \begin{enumerate}
    \item Position and orientation in 1-D
    \item Position and orientation in 2-D
    \item Position and orientation in 3-D
    \item Position and orientation in 4-D
  \end{enumerate} (10 marks. Estimated time: 15 min)
  Justify your answer.
\end{prob}

\ifsol
\paragraph*{Solution}
\begin{enumerate}
  \item 1D: position requires 1-DoF,  orientation requires a binary flag.
  \item 2D: position requires 2-DoF, orientation requires a single angle: 1-DoF.
  \item 3D: position requires 3-DoF, orientation requires 3-DoF because there
    are three pairs of axis X-Y, Y-Z, Z-X.
  \item 4D: position requires 4-DoF, orientation requires 6-DoF because there
    are 6 pairs of axis: $^4C_2 = 6$.
\end{enumerate}
\fi

\begin{prob}
  Write a program in C++ that checks if a given
  3x3 matrix is a valid Rotation matrix is a valid Rotation matrix  (check for
  orthonormality i.e. orthogonality and determinant = 1). You may use Eigen's
  matrix multiplication and determinant() function. (10 marks. Used in the
  following problems. Estimated time: 15 min). 
\end{prob}

\ifsol
\paragraph*{Solution}
Please look at the function \verb|is_valid_rot| in file \verb|hw1.cpp|.
\fi

\begin{prob}
  In class, we proved the expression to convert roll ($\theta$), pitch
  ($\phi$), yaw ($\psi$) from Euler Angles to Rotation matrix,
  \begin{align}
    R(\theta, \phi, \psi) =
    \begin{bmatrix}
      r_{11}& r_{12} & r_{13} \\
      r_{21}& r_{22} & r_{23} \\
      r_{31}& r_{32} & r_{33}
    \end{bmatrix} 
                       = R_z(\psi) R_y(\phi) R_x(\theta).
  \end{align}
  
  What if we want to do the inverse? Prove that given a proper 3x3 rotation
  matrix  ($R^\top R = I$ and $det(R) = 1$) , the Euler angles are given by
  \begin{align}
    \begin{bmatrix}
    \theta(R) \\
    \phi(R) \\
    \psi(R)
    \end{bmatrix}
    = \begin{bmatrix}
      \arctantwo(r_{32}, r_{33}) \\
      -\arcsin(r_{31}) \\
      \arctantwo(r_{21}, r_{11})
      \end{bmatrix}
  \end{align}
  where $r_{ij}$ is the element in $i$th row and $j$th column of the rotation
  matrix $R$. (10 marks. Used in the
  following problems. Estimated time: 15 min).
\end{prob}

\ifsol
\paragraph*{Solution}
\begin{align}
&R(\theta, \phi, \psi) = \begin{bmatrix}
  c_\psi & - s_\psi & 0 \\
  s_\psi & c_\psi & 0 \\
  0 & 0 & 1
  \end{bmatrix}
  \begin{bmatrix}
    c_\phi & 0 & s_\phi \\
    0 & 1 & 0 \\
    - s_\phi & 0 & c_\phi
    \end{bmatrix}
 \notag\\&\quad
    \begin{bmatrix}
      1 & 0 & 0 \\
      0 & c_\theta & - s_\theta \\
      0 & s_\theta & c_\theta
      \end{bmatrix}
\notag\\&\quad
  =\begin{bmatrix}
    c_\psi & - s_\psi & 0 \\
    s_\psi & c_\psi & 0 \\
    0 & 0 & 1
  \end{bmatrix}
            \begin{bmatrix}
              c_\phi & s_\phi s_\theta & s_\phi c_\theta \\
              0 & c_\theta & -s_\theta \\
              -s_\phi & c_\phi s_\theta & c_\phi c_\theta
            \end{bmatrix}
\notag\\&\quad
  = \begin{bmatrix}
    c_\psi c_\phi & c_\psi s_\phi s_\theta - s_\psi c_\theta & c_\psi s_\phi c_\theta+ s_\psi s_\theta \\
    s_\psi c_\phi & s_\psi s_\phi s_\theta + c_\psi c_\theta & s_\psi s_\phi c_\theta - c_\psi s_\theta \\
    -s_\phi & c_\phi s_\theta & c_\phi c_\theta
    \end{bmatrix}
\end{align}

Starting from the first term of R.H.S. of the equation to be proved,
%
\begin{mutline}
  \arctan2(r_{32}, r_{33})
  = \arctan2(c_\phi s_\theta, c_\phi c_\theta) = \tan^{-1}(\frac{c_\phi
    s_\theta}{c_\phi c_\theta}) = \tan^{-1}(\frac{s_\theta}{c_\theta}) =
  \tan^{-1}(\tan(\theta)) = \theta.
\end{mutline}
%
For the second term, $-\arcsin(r_{31}) = -\sin^{-1}(-s_\phi) =  \phi$.
For the third term ,
\begin{mutline}
  \arctan2(r_{21}, r_{11})
  = \arctan2(s_\psi c_\phi, c_\psi c_\phi) = \tan^{-1}(\frac{s_\psi
    c_\phi}{c_\psi c_\phi}) = \tan^{-1}(\frac{s_\psi}{c_\psi}) =
  \tan^{-1}(\tan(\psi)) = \psi.
\end{mutline}

\fi

\begin{prob}
  Write a pair of functions in C++ that converts rotation matrix from XYZ Euler
angles (roll, pitch, yaw) and vice versa. Test the pair of functions with
randomly generated Euler angles. And check if the converted rotation matrix is
orthonormal. What happens when pitch = $\pi/2$, are you able to convert from
rotation matrix to Euler angle? Why or why not? (50 marks. Estimated time: 30 min)
\label{prob:euler-to-rotmat}
\end{prob}

\ifsol
\paragraph*{Solution}
Please check function \verb|euler_to_rotmat| and \verb|rotmat_to_euler_rpy| in
\verb|hw1.cpp|. When pitch $=\frac{\pi}{2}$ works some of the time, but not all the time because of the
Gimbal lock matrix to Euler angle conversion has multiple possible solutions.
\fi

\begin{prob}
  Write a function in C++ that generates a 4x4 transformation matrix given XYZ Euler
  angles (roll, pitch, yaw) and translation. You can use the function that you
  wrote in Prob~\ref{prob:euler-to-rotmat}(20 marks. Estimated time: 15 min).
\end{prob}

\ifsol
\paragraph*{Solution}
Please check function \verb|transformation_matrix| in \verb|hw1.cpp|
\fi


%\bibliography{main}
%\bibliographystyle{plain}
\end{document}